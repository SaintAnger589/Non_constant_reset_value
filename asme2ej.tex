\documentclass[twocolumn,10pt]{asme2ej}

\usepackage{epsfig} %% for loading postscript figures

%% The class has several options
%  onecolumn/twocolumn - format for one or two columns per page
%  10pt/11pt/12pt - use 10, 11, or 12 point font
%  oneside/twoside - format for oneside/twosided printing
%  final/draft - format for final/draft copy
%  cleanfoot - take out copyright info in footer leave page number
%  cleanhead - take out the conference banner on the title page
%  titlepage/notitlepage - put in titlepage or leave out titlepage
%  
%% The default is oneside, onecolumn, 10pt, final


\title{Non constant reset value for the design}

%%% first author
\author{Gaurav Srivastava
    \affiliation{
	Seattle, Washington 98012\\
    Email: gauravsri589@gmail.com
    }	
}

\begin{document}

\maketitle    

%%%%%%%%%%%%%%%%%%%%%%%%%%%%%%%%%%%%%%%%%%%%%%%%%%%%%%%%%%%%%%%%%%%%%%
\begin{abstract}
{\it This paper illustrates the designing a circuit 
which has a non constant reset value. Normally the 
standard way of designing the Verilog circuit is to have a
constant value in the reset. The paper describes method to have
a non constant reset value in the design both for the single 
bit and multi-bit signal.
}
\end{abstract}

%%%%%%%%%%%%%%%%%%%%%%%%%%%%%%%%%%%%%%%%%%%%%%%%%%%%%%%%%%%%%%%%%%%%%%
\section{Introduction}
The paper describes non constant reset values for a register. 
Both the single-bit and multi-bit methods are elaborated. 
Spyglass run confirms that the above methods are valid and works correctly. 

\section{Problem Description}
Normally the standard rule is that the register has a constant static value on the reset. This is illustrated in the following example

\begin{verbatim}
module test(
  input clk,
  input rst_n,
  input val,
  output reg reg1
);

always @(posedge clk or negedge rst_n) 
begin
  if (!rst_n) 
  begin
    reg1 <= 1'b0;
  end
    reg1 <= val;
end
\end{verbatim}

As can be seen in the example, the reset value is a constant static value. This is common in designs and is standard. However, in some designs we require the reset values to be changed rather then the static value. The below code gives out error when the reset value is set to a non-constant value. 

\begin{verbatim}
module test(
  input clk,
  input rst_n,
  input reset_val,
  input val
  output reg reg1
);

always @(posedge clk or negedge rst_n) 
begin
  if (!rst_n) 
  begin
    reg1 <= reset_val;
  end
    reg1 <= val;
end
\end{verbatim}

The LINT in the above code will error out saying the reset value cannot be a non-constant value. This will give out different result in simulation and synthesis. 

There are 2 possible scenario:
\begin{itemize}
  \item[$\bullet$] Single bit register
  \item[$\bullet$] Multi bit register
\end{itemize}

The solution for designing this non-constant type circuit is described below. The workaround for the multi-bit register can also be applied for the single-bit register. However, the single bit method is faster.
%%%%%%%%%%%%%%%%%%%%%%%%%%%%%%%%%%%%%%%%%%%%%%%%%%%%%%%%%%%%%%%%%%%%%%
\section{Single bit Register method}  

Since the value change in the single bit can be 1 or 0, reset\_val can be combined with rst\_n in the following manner \cite{mis}

\begin{verbatim}
module test(
  input clk,
  input rst_n,
  input reset_val,
  input val
  output reg reg1
);


assign rst_int  = rst_n | reset_val;
assign prst_int = rst_n | !reset_val;

always @(posedge clk 
         or negedge rst_int 
         or negedge prst_int) 
begin
  if (!rst_n) 
  begin
    reg1 <= 1'b0;
  end else if (!prst_int)
  begin
    reg1 <= 1'b1;
  end
    reg1 <= val;
end
\end{verbatim}

%%%%%%%%%%%%%%%%%%%%%%%%%%%%%%%%%%%%%%%%%%%%%%%%%%%%%%%%%%%%%%%%%%%%%%
%%%%%%%%%%%%%%% begin table   %%%%%%%%%%%%%%%%%%%%%%%%%%

\begin{center}
\begin{tabular}{c c c c c c c}
\label {Truth table for the single bit non constant reset value}
& & \\ % put some space after the caption
\hline
rst\_n& reset\_val & rst\_int & prst\_int & reg\_value \\
\hline
0 & 1 & 1 & 0 & 1 \\
0 & 0 & 0 & 1 & 0 \\
\hline
\end{tabular}
\end{center}
%%%%%%%%%%%%%%%% end table %%%%%%%%%%%%%%%%%%% 

\section{Multi bit register method}
This method is based that the reset value can be assigned with a parameter
with a default value. 
This default value can be overridden at the instantiation level by passing
different values to the parameters. Let us consider an example

\begin{verbatim}
module top_inst1(
  input clk,
  input rst_n,
  input val
  output reg reg1
);

  test #(
    .RESET_VAL(32'h0f00)
  ) test1(
    .clk(clk),
    .rst_n(rst_n),
    .val(val),
    .reg1(reg1)
  );
  
endmodule

module top_inst2(
  input clk,
  input rst_n,
  input val
  output reg reg1
);

  test #(
    .RESET_VAL(32'h0bb0)
  ) test2(
    .clk(clk),
    .rst_n(rst_n),
    .val(val),
    .reg1(reg1)
  );
  
endmodule

module test #(
  parameter RESET_VAL = 0
)
(
  input clk,
  input rst_n,
  input val
  output reg reg1
);

always @(posedge clk or negedge rst_n) 
begin
  if (!rst_n) 
  begin
    reg1 <= RESET_VAL;
  end else 
  begin
    reg1 <= val;
  end
end
endmodule
\end{verbatim}


As can be seen, test is instantiated in 2 top modules 
with different parameters. The reset value is governed by the 
parameter RESET\_VAL whose values can be changed from the top level instance. 

This method is also useful when you want to derive the value based on some defines. 
The 2 top instances can have different define values, which will be then passed to the 
parameter. 

%%%%%%%%%%%%%%%%%%%%%%%%%%%%%%%%%%%%%%%%%%%%%%%%%%%%%%%%%%%%%%%%%%%%%%
\section{Conclusions}
As the paper described, we can have non constant reset values for a register. 
Both the single-bit and multi-bit methods are described. 
Spyglass run confirms that the above methods are valid and works correctly. 

%%%%%%%%%%%%%%%%%%%%%%%%%%%%%%%%%%%%%%%%%%%%%%%%%%%%%%%%%%%%%%%%%%%%%%
% The bibliography is stored in an external database file
% in the BibTeX format (file_name.bib).  The bibliography is
% created by the following command and it will appear in this
% position in the document. You may, of course, create your
% own bibliography by using thebibliography environment as in
%
%\begin{thebibliography}{12}
% ...
% \bibitem{itemreference} D. E. Knudsen.
% {\em 1966 World Bnus Almanac.}
% {Permafrost Press, Novosibirsk.}
% ...
% \end{thebibliography}

% Here's where you specify the bibliography style file.
% The full file name for the bibliography style file 
% used for an ASME paper is asmems4.bst.

 \bibliographystyle{asmems4}
% Here's where you specify the bibliography database file.
% The full file name of the bibliography database for this
% article is asme2e.bib. The name for your database is up
% to you.
 \bibliography{asme2e}

%%%%%%%%%%%%%%%%%%%%%%%%%%%%%%%%%%%%%%%%%%%%%%%%%%%%%%%%%%%%%%%%%%%%%%


\end{document}
